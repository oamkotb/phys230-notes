\documentclass[11pt]{article}

\usepackage[margin=1in]{geometry}
\usepackage{amsmath, amssymb}
\usepackage{float}
\usepackage{graphicx}
\usepackage{hyperref}
\usepackage[nopatch=footnote]{microtype}
\usepackage{physics}
\usepackage{setspace}
\usepackage{siunitx}
\usepackage{ulem}
\usepackage{wrapfig}

\AtBeginDocument{\RenewCommandCopy\qty\SI}
\graphicspath{{imgs/}}

\setstretch{1.1}
\setlength{\parindent}{0pt}
\setlength{\parskip}{6pt}

\begin{document}

\section{Electrostatics}

\subsection{Electric Field}

\subsubsection{Introduction}

The fundamental problem that electrodynamics seeks to answer is the following:
given a collection of electric charges \( q_1, q_2, q_3, \ldots \) (the \textit{source charges}), what force do they exert on another charge \( Q \) (the \textit{test charge})?

\begin{figure}[H]
    \centering
    \includegraphics[width=0.55\textwidth]{source_test_charges.png}
    \caption{Source charges exerting a force on a test charge.}
    \vspace{-0.5em}
\end{figure}

\textbf{Electrostatics:} Charges are \uline{at rest} relative to one another.

\textbf{Electrodynamics:} Charges are \uline{in motion} relative to one another.

\vspace{0.3em}

These problems are solvable due to the \textbf{principle of superposition}, which states that the \uline{interaction between any two charges is unaffected by the presence of other charges}. As a result, we can calculate the force due to each source charge independently and then add the results:

\[
    \vb{F} = \vb{F}_1 + \vb{F}_2 + \cdots
\]

While this principle simplifies the physics conceptually, real problems may still be complex because the positions, velocities, and accelerations of charges (both present and past) can matter. To build intuition, we begin with electrostatics.


\subsubsection{Coulomb's Law}

The force on a test charge \( Q \) due to a single point charge \( q \), at rest and separated by a distance \( r \), is given by \textbf{Coulomb’s Law}:

\[
\boxed{
    \vb{F} = \frac{1}{4\pi\varepsilon_0}
    \frac{qQ}{r^2}\,\hat{\vb{r}}
}
\]

Here, \( \varepsilon_0 \) is the \textbf{permittivity of free space}:

\[
    \varepsilon_0 = \qty{8.85e-12}{\coulomb^2\per\newton\meter^2}
\]

It is often convenient to define the constant

\[
    k = \frac{1}{4\pi\varepsilon_0}
      = \qty{9.0e9}{\newton\meter^2\per\coulomb^2}
\]

so Coulomb’s Law can be written more compactly as

\[
\boxed{
    \vb{F} = k\frac{qQ}{r^2}\,\hat{\vb{r}}
}
\]

\subsubsection{The Electric Field}

The \textbf{electric field} \( \vb{E} \) is a vector quantity defined at every point in space and depends on the configuration of source charges. Conceptually, it represents an entity that fills space around charges: when a test charge is placed at a given point, the electric field at that point determines the force it experiences.

If multiple point charges are present, the total force on a test charge \( Q \) is

\[
    \vb{F}
    = \vb{F}_1 + \vb{F}_2 + \cdots
    = k\left(
        \frac{q_1 Q}{r_{1Q}^2}\hat{\vb{r}}_{1Q}
      + \frac{q_2 Q}{r_{2Q}^2}\hat{\vb{r}}_{2Q}
      + \cdots
    \right)
\]

Factoring out \( Q \),

\[
    \vb{F} = Q\,\vb{E}
\]

where the electric field is defined as

\[
\boxed{
    \vb{E}(\vb{r})
    = k \sum_{i=1}^{n}
      \frac{q_i}{r_{iQ}^2}\,\hat{\vb{r}}_{iQ}
}
\]

Note that the electric field depends only on the source charges and the observation point, not on the test charge itself.

\subsubsection{Charge Distributions}

If the charge is distributed continuously over some region, the sum becomes an integral:
\[
\boxed{
    \vb{E(r)} = k\int\frac{1}{r^2_{qQ}}\vectorunit{r}_{qQ}\dd{q}
}
\]
In many physical situations, charge is distributed continuously rather than concentrated at a point. Such configurations are described using a \textbf{charge density}.

\begin{enumerate}

\item \textbf{Linear Charge Distribution}

Charge is distributed along a one-dimensional object (e.g., a thin wire) and is described by the \textbf{linear charge density}.

\begin{figure}[H]
    \centering
    \includegraphics[width=0.40\textwidth]{linear_distribution.png}
    \vspace{-0.5em}
\end{figure}

\[
\boxed{
    \lambda \equiv \dv{q}{\ell}
    \qquad \left[\si{\coulomb\per\meter}\right]
}
\]

For an infinitesimal segment of length \( \dd{\ell} \),
\[
    \dd{q} = \lambda\,\dd{\ell}
\]

\item \textbf{Surface Charge Distribution}

Charge is distributed over a two-dimensional surface and is described by the \textbf{surface charge density}.

\begin{figure}[H]
    \centering
    \includegraphics[width=0.25\textwidth]{surface_distribution.png}
    \vspace{-0.5em}
\end{figure}
\[
\boxed{
    \sigma \equiv \dv{q}{A}
    \qquad \left[\si{\coulomb\per\meter\squared}\right]
}
\]

For an infinitesimal surface element \( \dd{A} \),
\[
    \dd{q} = \sigma\,\dd{A}
\]

\item \textbf{Volume Charge Distribution}

Charge is distributed throughout a three-dimensional region and is described by the \textbf{volume charge density}.

\begin{figure}[H]
    \centering
    \includegraphics[width=0.15\textwidth]{volume_distribution.png}
    \vspace{-0.5em}
\end{figure}

\[
\boxed{
    \rho \equiv \dv{q}{V}
    \qquad \left[\si{\coulomb\per\meter\cubed}\right]
}
\]

For an infinitesimal volume element \( \dd{V} \),
\[
    \dd{q} = \rho\,\dd{V}
\]

\end{enumerate}

\textbf{Average Charge Density}

In cases where the charge distribution is not uniform or when only the total charge over a finite region is known, it is useful to define an \textbf{average charge density}.

\[
\boxed{
    \bar\lambda = \frac{q}{L}
    \qquad
    \bar\sigma = \frac{q}{A}
    \qquad
    \bar\rho = \frac{q}{V}
}
\]

where:
\begin{itemize}
    \item \( q \) is the total charge,
    \item \( L \) is the total length,
    \item \( A \) is the total surface area,
    \item \( V \) is the total volume.
\end{itemize}

If there is uniform charge distribution, the charge density at any location is equivalent to the average charge density ($\rho = \bar\rho$).

\hrulefill

\subsection{Divergence and Curl of Electrostatic Fields}
\subsubsection{Field Lines, Flux, and Gauss’s Law}
The rest of electrostatics is assembling tools and tricks to avoid the involved integrals. The first of these tools is electric field lines. An \textbf{electric field line} is an imaginary curve drawn through a region of space so that its tangent at any point is in the direction of the electric field vector at that point.

\begin{figure}[H]
    \centering
    \includegraphics[width=0.8\textwidth]{field_lines.png}
    \caption{Field line for two oppsoite charges and two equal charges.}
    \vspace{-0.5em}
\end{figure}

The magnitude of the field is indicated by the density of the field lines: it's stronger near the center point charge where the field lines are closer together. Field lines never intersect, because E has a unique dierction at each point. In terms of direction of the arrows, \uline{field lines begin on positive charges and end on negative ones}. Field lines also cannot terminate in midair, though they may extend out to infinity.

For a given (arbitrary) density of electric field lines, the \textbf{flux} of the electric field $\vb{E}$ through a surface $\mathcal{S}$ is proportional to the number of field lines passing through the surface. This follows because the field strength is proportional to the density of field lines (the number per unit area). Hence, $\vb{E} \cdot \dd{\vb{A}}$ is proportional to the number of field lines passing through the infinitesimal area $\dd{\vb{A}}$. The dot product selects the component of $\vb{E}$ perpendicular to the surface:
\[
    \Phi_{\vb{E}} \equiv \int_{\mathcal{S}} \vb{E} \cdot \dd{\vb{A}}
\]

\begin{figure}[H]
    \centering
    \includegraphics[width=0.3\textwidth]{flux_on_s.png}
    \caption{Flux is proportional to the number of electric field lines passing through a surface.}
    \vspace{-0.5em}
\end{figure}

Since only the component of $\vb{E}$ perpendicular to the surface contributes to the flux, we may write (for a field of constant magnitude over the surface)
\[
    \Phi_{\vb{E}} = E \int_{\mathcal{S}} \cos\theta \, \dd{A}
\]
where $\theta$ is the angle between $\vb{E}$ and the area vector $\dd{\vb{A}}$.

\begin{figure}[H]
    \centering
    \includegraphics[width=0.3\textwidth]{ortho_to_e_flux.png}
    \caption{Angle between $\vb{E}$ and the surface element $\dd{\vb{A}}$.}
    \vspace{-0.5em}
\end{figure}

The electric flux through any closed surface reflects the total charge contained inside the surface. This is because electric field lines due to charges \emph{outside} the surface pass into and out of it in equal amounts, giving no net contribution. Only field lines that originate from positive charges inside the surface or terminate on negative charges inside the surface contribute to a net flux. This is the essence of \textbf{Gauss's Law}.

To derive a quantitative explanation, consider a point charge $q$ located at the origin. The electric field produced by the charge is
\[
    \vb{E} = \frac{1}{4\pi\epsilon_0} \frac{q}{r^2} \vu{r}.
\]

The electric flux through a spherical surface of radius $r$ centered on the charge is
\[
    \Phi_{\vb{E}} = \oint \vb{E} \cdot \dd{\vb{a}}
    = \int \frac{1}{4\pi\epsilon_0} \left( \frac{q}{r^2} \vu{r} \right)
    \cdot \left( r^2 \sin\theta \dd{\theta} \dd{\phi} \, \vu{r} \right).
\]

Since $\vu{r} \cdot \vu{r} = 1$, this simplifies to
\[
    \Phi_{\vb{E}} = \frac{q}{4\pi\epsilon_0}
    \int_0^{2\pi} \int_0^\pi \sin\theta \dd{\theta} \dd{\phi}
    = \frac{q}{4\pi\epsilon_0} (4\pi)
    = \frac{q}{\epsilon_0}.
\]

\begin{figure}[H]
    \centering
    \includegraphics[width=0.3\textwidth]{charge_in_sphere.png}
    \caption{Point charge $q$ at the origin within an enclosed surface.}
    \vspace{-0.5em}
\end{figure}

Notice that as the surface area increases as $r^2$, the magnitude of the electric field decreases as $1/r^2$. In terms of field lines, this means that the same number of field lines pass through any spherical surface enclosing the charge. Consequently, \emph{the flux through any closed surface enclosing the charge is} $\displaystyle \frac{q}{\epsilon_0}$, independent of the size or shape of the surface.

Now consider a collection of $n$ point charges enclosed by a closed surface. By the principle of superposition, the total electric field is
\[
    \vb{E} = \sum_{i=1}^{n} \vb{E}_i.
\]

The total flux through the surface is therefore
\[
    \oint \vb{E} \cdot \dd{\vb{a}}
    = \sum_{i=1}^{n} \oint \vb{E}_i \cdot \dd{\vb{a}}
    = \sum_{i=1}^{n} \frac{q_i}{\epsilon_0}
    = \frac{Q_{\text{enc}}}{\epsilon_0}.
\]

Thus, for \emph{any} closed surface,
\[
\boxed{
    \oint \vb{E} \cdot \dd{\vb{a}} = \frac{Q_{\text{enc}}}{\epsilon_0}
}
\]

This is \textbf{Gauss’s Law}. It can be used in three primary ways:
\begin{enumerate}
    \item If the enclosed charge $Q_{\text{enc}}$ is known, the electric flux can be found directly.
    \item If $Q_{\text{enc}}$ is known and sufficient symmetry exists, the electric field $\vb{E}$ can be determined.
    \item If the electric field $\vb{E}$ is known, the enclosed charge can be found by integration.
\end{enumerate}
\hrulefill

\section{Potentials}
\subsection{Electric Dipoles}

An electric dipole consists of a pair of point charges with:
\begin{itemize}
	\item equal magnitude of charge $q$,
	\item opposite signs,
	\item separated by a distance $d$.
\end{itemize}

\begin{figure}[H]
    \centering
    \includegraphics[width=0.2\textwidth]{dipoles.png}
    \caption{Diagram of an electric dipole.}
    \vspace{-0.5em}
\end{figure}

An electric dipole is defined by the \uline{dipole moment vector}
\[
\boxed{
    \vb{p} = q \vb{r}_{-+}
}
\]
where
\[
    \vb{r}_{-+} = \vb{r}_{-} - \vb{r}_{+}, \qquad
    d = |\vb{r}_{-+}|.
\]
The magnitude of the dipole moment is therefore
\[
    p = qd.
\]

Dipole moments are vectors and therefore add vectorially. If two dipoles have moments $\vb{p}_1$ and $\vb{p}_2$, the total dipole moment is $\vb{p}_1 + \vb{p}_2$. This allows closely spaced charges to be approximated as dipoles, simplifying calculations.

Although a water molecule is electrically neutral overall, the arrangement of its chemical bonds causes a displacement of charge. This results in a partial negative charge near the oxygen atom and partial positive charges near the hydrogen atoms, forming an electric dipole.

\begin{figure}[H]
    \centering
    \includegraphics[width=0.4\textwidth]{h2o_dipole.png}
    \caption{O--H bonds in a water molecule form an electric dipole.}
    \vspace{-0.5em}
\end{figure}

\subsubsection{Forces and Torque on Electric Dipoles}

The forces $\vb{F}_{+}$ and $\vb{F}_{-}$ acting on the charges of a dipole in an external electric field are equal in magnitude and opposite in direction, but they do not act along the same line. As a result, the net force on the dipole is zero (in a uniform field), but a \textbf{torque} acts on the dipole.

\begin{figure}[H]
    \centering
    \includegraphics[width=0.4\textwidth]{dipole_in_field.png}
    \caption{The net force on an electric dipole in a uniform field is zero, but a torque acts to rotate the dipole.}
    \vspace{-0.5em}
\end{figure}

Using the definition of torque,
\begin{align*}
  \boldsymbol{\tau}
  &= \boldsymbol{\tau}_{+} + \boldsymbol{\tau}_{-} \\
  &= \vb{r}_{+} \times \vb{F}_{+} + \vb{r}_{-} \times \vb{F}_{-} \\
  &= \vb{r}_{+} \times q\vb{E} + \vb{r}_{-} \times (-q)\vb{E} \\
  &= q(\vb{r}_{-} - \vb{r}_{+}) \times \vb{E} \\
  &= \vb{p} \times \vb{E}.
\end{align*}

Thus,
\[
\boxed{
    \boldsymbol{\tau} = \vb{p} \times \vb{E}
}
\]

The magnitude of the torque is
\[
    |\boldsymbol{\tau}| = pE \sin\phi,
\]
where $\phi$ is the angle between $\vb{p}$ and $\vb{E}$.

\begin{figure}[H]
    \centering
    \includegraphics[width=0.3\textwidth]{dipole_torque_rhr.png}
    \caption{The direction of $\boldsymbol{\tau}$ is determined using the right-hand rule.}
    \vspace{-0.5em}
\end{figure}

\subsubsection{Potential Energy of Electric Dipoles}

The potential energy of a dipole with moment $\vb{p}$ placed in a uniform external electric field $\vb{E}$ is defined as the negative of the work done by the field on the dipole:
\[
    \dd{U} = -\dd{W}.
\]

The infinitesimal work done by the electric field is
\begin{align*}
    \dd{W}
    &= \dd{W}_{+} + \dd{W}_{-} \\
    &= \vb{F}_{+} \cdot \dd{\vb{r}}_{+} + \vb{F}_{-} \cdot \dd{\vb{r}}_{-} \\
    &= q\vb{E} \cdot \dd{\vb{r}}_{+} - q\vb{E} \cdot \dd{\vb{r}}_{-} \\
    &= q\vb{E} \cdot (\dd{\vb{r}}_{+} - \dd{\vb{r}}_{-}).
\end{align*}

Assuming $\vb{E}$ is constant,
\begin{align*}
    W
    &= q\vb{E} \cdot (\vb{r}_{+} - \vb{r}_{-}) \\
    &= q\vb{E} \cdot \vb{r}_{-+} \\
    &= \vb{p} \cdot \vb{E}.
\end{align*}

Therefore, the potential energy of the electric dipole is
\[
\boxed{
    U = -\vb{p} \cdot \vb{E}.
}
\]
\end{document}
